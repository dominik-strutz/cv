%%%%%%%%%%%%%%%%%%%%%%%%%%%%%%%%%%%%%%%%%%%%%%%%%%%%%%%%%%%%%%%%%%%%%%%%%%%%%%%
% A clean template for an academic CV
%
% Uses tabularx to create two column entries (date and job/edu/citation).
% Defines commands to make adding entries simpler.
%
%%%%%%%%%%%%%%%%%%%%%%%%%%%%%%%%%%%%%%%%%%%%%%%%%%%%%%%%%%%%%%%%%%%%%%%%%%%%%%%

\documentclass[10pt, a4paper]{article}

% Full Unicode support for non-ASCII characters
\usepackage[utf8]{inputenc}

\usepackage{xspace} % to get the spacing after macros right

% Useful aliases
\newcommand{\UOE}{University of Edinburgh\xspace}
\newcommand{\LMU}{Ludwig-Maximilians-Universität\xspace}
\newcommand{\TUM}{Technische Universität München\xspace}
\newcommand{\SoG}{School of Geosciences\xspace}

% Identifying information
\newcommand{\Title}{Curriculum Vit\ae}
\newcommand{\FirstName}{Dominik}
\newcommand{\LastName}{Strutz}
\newcommand{\Initials}{D}
\newcommand{\MyName}{\FirstName\ \LastName}
\newcommand{\Me}{\textbf{\LastName, \Initials}}  % For citations
\newcommand{\Email}{domink.strutz@ed.ac.uk}
\newcommand{\PersonalWebsite}{dominik-strutz.github.io/}
\newcommand{\LabWebsite}{blogs.ed.ac.uk/curtis/}
% \newcommand{\ORCID}{0000-0001-6123-9515}
% \newcommand{\Address}{
%   Jane Herdman Building \\ 4 Brownlow Street \\ Liverpool, L69 3GP \\ United Kingdom
% }

% Names for citing coauthors
\newcommand{\AC}{Curtis, A}

% Template configuration
%%%%%%%%%%%%%%%%%%%%%%%%%%%%%%%%%%%%%%%%%%%%%%%%%%%%%%%%%%%%%%%%%%%%%%%%%%%%%%%

% Disable hyphenation
\usepackage[none]{hyphenat}

% Control the font size
\usepackage{anyfontsize}

% Icon fonts (requires using xelatex or luatex)
\usepackage[fixed]{fontawesome5}
\usepackage{academicons}

% Template variables for styling
\newcommand{\TablePad}{\vspace{-0.4cm}}
\newcommand{\SoftwareTitle}[1]{{\bfseries #1}}
\newcommand{\TableTitle}[1]{{\fontsize{12pt}{0}\selectfont \itshape #1}}

% For fancy and multipage tables
\usepackage{tabularx}
\usepackage{ltablex}

% Define a new environment to place all CV entries in a 2-column table.
% Left column are the dates, right column the entries.
\usepackage{environ}
\NewEnviron{EntriesTable}{
\TablePad
\begin{tabularx}{\textwidth}{@{}p{0.10\textwidth}@{\hspace{0.02\textwidth}}p{0.88\textwidth}@{}}
  \BODY
\end{tabularx}
}
\NewEnviron{EntriesTableExtra}{
\TablePad
\begin{tabularx}{\textwidth}{@{}p{0.10\textwidth}@{\hspace{0.02\textwidth}}p{0.79\textwidth}@{\hspace{0.02\textwidth}}>{\raggedright\arraybackslash}p{0.07\textwidth}}
  \BODY
\end{tabularx}
}

% Macros to add links and mark publications
\newcommand{\DOI}[1]{doi:\href{https://doi.org/#1}{#1}}
\newcommand{\DOILink}[1]{\href{https://doi.org/#1}{doi.org/#1}}
\newcommand{\Website}[1]{\href{https://#1}{#1}}
\newcommand{\Preprint}[1]{\href{https://doi.org/#1}{\faFilePdf}}
\newcommand{\Youtube}[1]{\href{https://www.youtube.com/watch?v=#1}{\faYoutube}}
\newcommand{\GitHub}[1]{\href{https://github.com/#1}{\faGithub}}
\newcommand{\Data}[1]{\href{https://doi.org/#1}{\faChartLine}}
\newcommand{\Slides}[1]{\href{https://#1}{\faTv}}
\newcommand{\SlidesDOI}[1]{\href{https://doi.org/#1}{\faTv}}
\newcommand{\PosterDOI}[1]{\href{https://doi.org/#1}{\faImage}}
\newcommand{\OA}{\thinspace\aiOpenAccess\enspace}

% Macros to set the year and duration on the left column
\newcommand{\Duration}[2]{\fontsize{9pt}{0}\selectfont #1 -- #2}
\newcommand{\Year}[1]{\fontsize{9pt}{0}\selectfont #1}
\newcommand{\Ongoing}{on}
\newcommand{\Future}{future}
\newcommand{\Appointment}[4]{\textbf{#1} \newline #2 \newline #3 \newline #4}

% Define command to insert month name and year as date
\usepackage{datetime}
\newdateformat{monthyear}{\monthname[\THEMONTH], \THEYEAR}

% Set the page margins
\usepackage[a4paper,margin=1.5cm,includehead,headsep=5mm]{geometry}

% To get the total page numbers (\pageref{LastPage})
\usepackage{lastpage}

% No indentation
\setlength\parindent{0cm}

% Increase the line spacing
\renewcommand{\baselinestretch}{1.2}
% and the spacing between rows in tables
\renewcommand{\arraystretch}{1.5}

% Remove space between items in itemize and enumerate
\usepackage{enumitem}
\setlist{nosep}

% Use custom colors
\usepackage[usenames,dvipsnames]{xcolor}

% Set fonts (requires compilation with xelatex)
\usepackage{fontspec}
\setmainfont[%
  Path = fonts/notoserif/,
  UprightFont = NotoSerif-Regular,
  BoldFont = NotoSerif-Bold,
  ItalicFont = NotoSerif-Italic,
  Extension = .ttf
]{NotoSerif}



% Set the spacing for sections
\usepackage{titlesec}
\titleformat{\section}
  {\normalfont\Large\mdseries} % format
  {} % label
  {0pt} % separation (left separation for hang)
  {} % text before title
  [\titlerule] % text after title
\titleformat{\subsection}
  {\normalfont\large\mdseries} % format
  {} % label
  {0pt} % separation (left separation for hang)
  {} % text before title

% Disable number of sections. Use this instead of "section*" so that the sections still
% appear as PDF bookmarks. Otherwise, would have to add the table of contents entries
% manually.
\makeatletter
\renewcommand{\@seccntformat}[1]{}
\makeatother

% Set fancy headers
\usepackage{fancyhdr}
\pagestyle{fancy}
\fancyhf{}
\lhead{\fontsize{9pt}{10pt}\selectfont
  \monthyear\today
}
\chead{
  \fontsize{9pt}{10pt}\selectfont
  \MyName
  \hspace{0.2cm} -- \hspace{0.2cm}
  \Title
}
% \rhead{\fontsize{9pt}{10pt}\selectfont \thepage{} of \pageref*{LastPage}}
\rhead{\fontsize{9pt}{10pt}\selectfont \thepage{}}

\renewcommand{\headrulewidth}{0pt}

% Metadata for the PDF output and control of hyperlinks
\usepackage[colorlinks=true]{hyperref}
\hypersetup{
  pdftitle={\MyName\ - \Title},
  pdfauthor={\MyName},
  linkcolor=blue,
  citecolor=blue,
  filecolor=black,
  urlcolor=MidnightBlue
}
%%%%%%%%%%%%%%%%%%%%%%%%%%%%%%%%%%%%%%%%%%%%%%%%%%%%%%%%%%%%%%%%%%%%%%%%%%%%%%%


\begin{document}

% No header for the first page
\thispagestyle{empty}

%%%%%%%%%%%%%%%%%%%%%%%%%%%%%%%%%%%%%%%%%%%%%%%%%%%%%%%%%%%%%%%%%%%%%%%%%%%%%%%
\begin{minipage}[t]{0.7\textwidth}
{\fontsize{22pt}{0}\selectfont\MyName}
\end{minipage}
\begin{minipage}[t]{0.3\textwidth}
  \begin{flushright}
    Last updated: \monthyear\today
  \end{flushright}
\end{minipage}
\\[-0.1cm]
\rule{\textwidth}{2pt}
\\[0.1cm]
\begin{minipage}[t]{0.7\textwidth}
    % ORCID: \href{https://orcid.org/\ORCID}{\ORCID}
    % \\
    Email: \href{mailto:\Email}{\Email}
    \\
    Research group: \Website{\LabWebsite}
    \\
    Website: \Website{\PersonalWebsite}
\end{minipage}
% \begin{minipage}[t]{0.3\textwidth}
%   \begin{flushright}
%     \Address
%   \end{flushright}
% \end{minipage}

%%%%%%%%%%%%%%%%%%%%%%%%%%%%%%%%%%%%%%%%%%%%%%%%%%%%%%%%%%%%%%%%%%%%%%%%%%%%%%%
% \section{Professional Appointments}

% \begin{EntriesTable}
%   \Duration{2019}{\Ongoing}  &
%   \Appointment{Lecturer}{\LIVEARTH}{\LIVENV}{\LIV, UK}
%   \\
%   \Duration{2018}{\Ongoing}  &
%   \Appointment{Affiliate Researcher}{\UHEARTH}{\SOEST}{\UHM, USA}
% \end{EntriesTable}


%%%%%%%%%%%%%%%%%%%%%%%%%%%%%%%%%%%%%%%%%%%%%%%%%%%%%%%%%%%%%%%%%%%%%%%%%%%%%%%
\section{Education}

\begin{EntriesTable}
  \Duration{2021}{\Ongoing}  &
  \textbf{PhD in Geophysics}, \UOE
  \\
  \Duration{2019}{2021}  &
  \textbf{MSc in Geophysics}, \LMU and \TUM
  \\
  \Duration{2017}{2019}  &
  \textbf{BSc in Earth Sciences}, \LMU and \TUM
  \\
  \Duration{2016}{2017}  &
  \textbf{BSc in Physics}, \TUM
\end{EntriesTable}


%%%%%%%%%%%%%%%%%%%%%%%%%%%%%%%%%%%%%%%%%%%%%%%%%%%%%%%%%%%%%%%%%%%%%%%%%%%%%%%
% \section{Grants \& Fellowships}

%%%%%%%%%%%%%%%%%%%%%%%%%%%%%%%%%%%%%%%%%%%%%%%%%%%%%%%%%%%%%%%%%%%%%%%%%%%%%%%
% \section{Academic Service}

%%%%%%%%%%%%%%%%%%%%%%%%%%%%%%%%%%%%%%%%%%%%%%%%%%%%%%%%%%%%%%%%%%%%%%%%%%%%%%%
% \section{Awards \& Honors}

%%%%%%%%%%%%%%%%%%%%%%%%%%%%%%%%%%%%%%%%%%%%%%%%%%%%%%%%%%%%%%%%%%%%%%%%%%%%%%%
% \section{Teaching}

% \subsection{Undergraduate}

% \begin{EntriesTableExtra}
%   \Duration{2020}{\Ongoing}  &
%   ENVS398: Global Geophysics and Geodynamics
%   \newline
%   Teaching lithosphere dynamics (50\% of module)
%   \newline
%   Module coordinator from 2021
%   \newline
%   \textit{\LIV}
%   & ~
% \end{EntriesTableExtra}

% \subsection{Workshops \& Short Courses}

% \begin{EntriesTableExtra}
% \Year{2022}  &
%   Crafting beautiful maps with PyGMT.
%   \textit{EGU 2022}
%   &
%   \GitHub{GenericMappingTools/egu22pygmt}
%   \\
%   ~ &
%   A geophysical tour of mid-ocean ridges.
%   \textit{Transform 2022} (online)
%   &
%   \GitHub{leouieda/transform2022}
%   \Youtube{NzJmRlJCNbQ}
%   \\
% \Year{2021} &
% \end{EntriesTableExtra}


%%%%%%%%%%%%%%%%%%%%%%%%%%%%%%%%%%%%%%%%%%%%%%%%%%%%%%%%%%%%%%%%%%%%%%%%%%%%%%%
% \section{Student supervision}

%%%%%%%%%%%%%%%%%%%%%%%%%%%%%%%%%%%%%%%%%%%%%%%%%%%%%%%%%%%%%%%%%%%%%%%%%%%%%%%
% \section{Media \& Outreach}

%%%%%%%%%%%%%%%%%%%%%%%%%%%%%%%%%%%%%%%%%%%%%%%%%%%%%%%%%%%%%%%%%%%%%%%%%%%%%%%
\section{Publications}

% \subsection{Peer-reviewed Papers}

% \begin{EntriesTableExtra}
% \Year{2021}  &
%   \Santiago, \Me.
%   Gradient-boosted equivalent sources.
%   \emph{Geophysical Journal International}.
%   \DOI{10.1093/gji/ggab297}.
%   &
%   \GitHub{compgeolab/eql-gradient-boosted}
%   \Preprint{10.31223/X58G7C}
%   \\
% \end{EntriesTableExtra}


% \subsection{Peer-reviewed Conference Proceedings}

% \begin{EntriesTableExtra}
% \Year{2014}  &
%   \Figura, \Val, \Me, \Bi, \JB.
%   A Single Euler Solution Per Anomaly,
%   \emph{76th EAGE Conference and Exhibition 2014},
%   \DOI{10.3997/2214-4609.20140891}.
%   & ~
%   \\
% \end{EntriesTableExtra}

% \subsection{Non-peer-reviewed Papers}

% \begin{EntriesTableExtra}
% \Year{2017}  &
%   \Me.
%   Step-by-step NMO correction,
%   \emph{The Leading Edge},
%   \DOI{10.1190/tle36020179.1}.
%   &
%   \OA
%   \GitHub{pinga-lab/nmo-tutorial}
%   \\
% \end{EntriesTableExtra}

\subsection{Preprints}

\begin{EntriesTableExtra}
\Year{2019}  &
  \Me, \AC.
  Variational Bayesian experimental design for geophysical applications
  \emph{arxiv}.
  \DOI{arXiv:2307.01039}
  &
  \OA
\end{EntriesTableExtra}

%%%%%%%%%%%%%%%%%%%%%%%%%%%%%%%%%%%%%%%%%%%%%%%%%%%%%%%%%%%%%%%%%%%%%%%%%%%%%%%
\section{Presentations}

% \subsection{Invited \& Keynotes}

\subsection{Other Presentations}

\begin{EntriesTableExtra}

\Year{2023} &
  \Me, \AC.
  Variational Experimental Design Methods for Geophysical Applications
  \emph{EGU General Assembly 2023}
  \\

\Year{2022}  &
  \Me, \AC.
  Bayesian Optimal Experimental Design for Geophysical Applications
  \emph{IPGP Seismology Seminars}
  \\

\Year{2022}  &
  Schuberth, B., \Me, and Schneider, A.
  Earth's free-oscillation spectrum as a tool to assess mantle circulation models
  \emph{EGU General Assembly 2022},
  \\
\end{EntriesTableExtra}


%%%%%%%%%%%%%%%%%%%%%%%%%%%%%%%%%%%%%%%%%%%%%%%%%%%%%%%%%%%%%%%%%%%%%%%%%%%%%%%
\section{Open Science}

\subsection{Open-source Software}

\begin{EntriesTable}
  \Duration{2019}{2020} &
  \textbf{Obspy} | \Website{docs.obspy.org}
  \newline
  Project dedicated to provide a Python framework for processing seismological data.
  \newline
  Role: fixing bugs, contributing to tools for array analysis
  \\
\end{EntriesTable}

% \subsection{FAIR Data}

%%%%%%%%%%%%%%%%%%%%%%%%%%%%%%%%%%%%%%%%%%%%%%%%%%%%%%%%%%%%%%%%%%%%%%%%%%%%%%%
\section{Miscellaneous}

% \subsection{Professional society membership}

\subsection{Languages}

\TablePad
\begin{tabularx}{\textwidth}{@{}p{0.15\textwidth} p{0.85\textwidth}@{}}
  German & Native
  \\
  English & proficient
  \\
  Swedish & elementary
\end{tabularx}

%%%%%%%%%%%%%%%%%%%%%%%%%%%%%%%%%%%%%%%%%%%%%%%%%%%%%%%%%%%%%%%%%%%%%%%%%%%%%%%
% \section{Glossary}

% These are the meanings of the symbols used throughout this document:
% \\
% \TablePad
% \begin{tabularx}{\textwidth}{@{}p{0.03\textwidth} p{0.97\textwidth}@{}}
%   \aiOpenAccess & Indicates that a publication is open-access
%   \\
%   \faGithub & Link to a code repository on GitHub
%   \\
%   \faFilePdf & Link to an open-access PDF, usually a preprint or postprint
%   \\
%   \faYoutube & Link to a video on YouTube
%   \\
%   \faChartLine & Link to a data archive
%   \\
%   \faTv & Link to presentation slides
%   \\
%   \faImage & Link to a poster
% \end{tabularx}

\end{document}
